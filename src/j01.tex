\documentclass[12pt]{article}
\usepackage[papersize={8cm,14cm},margin={.5cm,.5cm}]{geometry}
\usepackage{common}
\usepackage{pgfplots}
\begin{document}
\begin{problem}
  \item[1.] 一物體在直線上運動,圖(一)為其速度($\mathrm{v}$)與時間($\mathrm{t}$)的關係圖。在 $\mathrm{t} = \qty{0}{s}$ 至 $\mathrm{t} = \qty{5}{s}$ 的期間,物體的位移為何?
  \begin{figure}[ht]
    \centering
    \begin{tikzpicture}
      \begin{axis}[
        width=5cm,axis lines=left,axis line style={-latex},
        xlabel={t\,(\unit{s})},
        ylabel={v\,(\unit{m/s})},
        xmin=0,xmax=6,ymin=0,ymax=12,
        xtick={5},ytick={0,10},
        every x tick/.style={black,thick},
        every y tick/.style={black,thick},
        grid=none,
        xlabel style={at={(axis description cs:1,0)},anchor=north},
        ylabel style={at={(axis description cs:.3,1)},rotate=-90,align=center,anchor=south}
      ]
        \addplot[domain=0:6,thick] {-2*x+10};
      \end{axis}
    \end{tikzpicture}
    \vspace*{-2ex}
    \caption*{圖(一)}
    \vspace*{-2ex}
  \end{figure}
  \begin{choices}
    \item \qty{0.5}{m}
    \item \qty{2}{m}
    \item \qty{25}{m}
    \item \qty{100}{m}
  \end{choices}
\end{problem}
\end{document}

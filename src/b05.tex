\documentclass[12pt]{article}
\usepackage[papersize={8cm,12cm},margin={.5cm,.5cm}]{geometry}
\usepackage{common}
\newcommand{\particle}[1]{\tikz[baseline=-.85ex]{\draw[fill=#1] (0,0) circle (.15cm);}}
\begin{document}
\begin{problem}
  \item[5.] 表(四)列有鋁離子(\ch{Al^3+})中所含有的三種粒子的個數,其中 \particle{black}、\particle{black!25}、\particle{black!0} 可能為質子、中子或電子。根據表中資訊判斷,下列何者正確?
  \begin{table}[ht]
    \centering
    \renewcommand{\arraystretch}{1.2}
    \vspace*{-1ex}
    \caption*{表(四)}
    \vspace*{-1ex}
    \begin{tabular}{|c|c|}
      \hline
      粒子 & 個數 \\ \hline
      \particle{black} & 10 \\ \hline
      \particle{black!25} & 13 \\ \hline
      \particle{white} & 14 \\ \hline
    \end{tabular}
  \end{table}
  \begin{choices}
    \item 粒子 \particle{black} 是電子
    \item 粒子 \particle{white} 是質子
    \item 粒子 \particle{black} 位於原子核內
    \item 粒子 \particle{white} 位於原子核外
  \end{choices}
\end{problem}
\end{document}

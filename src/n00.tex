\documentclass[12pt]{article}
\usepackage[papersize={8cm,10cm},margin={.5cm,.5cm}]{geometry}
\usepackage{common}
\usepackage{amssymb}
\usepackage{xcolor}
\newcommand{\x}{\colorbox{red!30}{\rule[-.2\baselineskip]{0pt}{.9\baselineskip}$x$}}
\newcommand{\y}{\colorbox{blue!30}{\rule[-.2\baselineskip]{0pt}{.9\baselineskip}$y$}}
\newcommand{\coeffa}{\colorbox{orange!30}{\rule[-.2\baselineskip]{0pt}{.9\baselineskip}$a$}}
\newcommand{\coeffb}{\colorbox{teal!30}{\rule[-.2\baselineskip]{0pt}{.9\baselineskip}$b$}}
\newcommand{\coeffc}{\colorbox{teal!30}{\rule[-.2\baselineskip]{0pt}{.9\baselineskip}$c$}}
\begin{document}
\small
\begin{problem}[label={\Alph*.},widest=A,parsep=0ex]
  \item 點和直線的關係的例子:
  \begin{enumerate}[label=(\arabic*),left=0pt]
    \item 點 $(\textcolor{red!90!black}{3}, \textcolor{blue}{4})$ 在 $\x = 3$ 的圖形上,但是不在 $\x = 4$ 的圖形上。
    \item 點 $(\textcolor{red!90!black}{3}, \textcolor{blue}{4})$ 在 $\y = 4$ 的圖形上,但是不在 $\y = 3$ 的圖形上。
    \item 點 $(\textcolor{red!90!black}{3}, \textcolor{blue}{4})$ 在 $\x + \y = 7$ 的圖形上,但是不在 $\x + \y = 5$ 的圖形上。
  \end{enumerate}
  \item 我們會用 $y = \coeffa x + \coeffb$ 表示一般(不是鉛直線)的直線方程式。
  \item 我們會用 $x = \coeffc$ 表示鉛直線的方程式。
\end{problem}
\end{document}

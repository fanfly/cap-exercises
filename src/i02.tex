\documentclass[12pt]{article}
\usepackage[papersize={8cm,12cm},margin={.5cm,.5cm}]{geometry}
\usepackage{common}
\usepackage{xeCJKfntef}
\begin{document}
\begin{problem}
  \item[2.] 一間商店將巧克力包裝為方形、圓形禮盒出售,每盒方形禮盒的價格相同,每盒圓形禮盒的價格也相同。\CJKunderline{小劉}身上帶有一些錢,他發現如果購買 3 盒方形禮盒與 7 盒圓形禮盒的話,身上的錢會不足 240 元;如果購買 7 盒方形禮盒與 3 盒圓形禮盒的話,身上的錢會剩下 240 元。如果\CJKunderline{小劉}購買 10 盒方形禮盒的話,身上的錢會剩下幾元?
  \begin{choices}
    \item 360
    \item 480
    \item 600
    \item 720
  \end{choices}
\end{problem}
\end{document}

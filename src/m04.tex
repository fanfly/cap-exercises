\documentclass[12pt]{article}
\usepackage[papersize={8cm,14cm},margin={.5cm,.5cm}]{geometry}
\usepackage{common}
\usepackage{xeCJKfntef}
\usepackage{pgfplots}
\begin{document}
\begin{problem}
  \item[4.] 一箱中有紅、綠、藍三種顏色的紙牌,圖(一)為各顏色紙牌張數的長條圖。若\CJKunderline{小潘}從箱中抽出一張牌,且每張牌被抽出的機會相等,則\CJKunderline{小潘}抽出紅色紙牌的機率為何?
  \begin{figure}[ht]
    \centering
    \vspace*{-1ex}
    \begin{tikzpicture}
      \begin{axis}[
        width=6cm,height=4cm,axis lines=left,x axis line style=-,y axis line style={-latex},
        xlabel={紙牌顏色},
        ylabel={紙\\[-.3em]牌\\[-.3em]張\\[-.3em]數\\[-1.1em]\rotatebox{-90}{\symbol{40}}\\[-.1em]張\\[-1.2em]\rotatebox{-90}{\symbol{41}}},
        xmin=-1,xmax=3,ymin=0,ymax=4,
        xticklabels={紅,綠,藍},
        xtick=data,ytick={0,1,2,3},
        ymajorgrids,
        every x tick/.style={draw=none},every y tick/.style={black,thick},
        xlabel style={at={(axis description cs:.5,-.03)},anchor=north},
        ylabel style={at={(axis description cs:.1,1)},rotate=-90,align=center,anchor=north}
      ]
        \addplot[ybar,fill=black!40] coordinates {
          (0,3) (1,2) (2,1)
        };
      \end{axis}
    \end{tikzpicture}
    \vspace*{-1ex}
    \caption*{圖(一)}
    \vspace*{-2ex}
  \end{figure}
  \begin{choices}
    \item $\fraction{2}{3}$
    \item $\fraction{1}{2}$
    \item $\fraction{1}{3}$
    \item $\fraction{1}{6}$
  \end{choices}
\end{problem}
\end{document}

\documentclass[12pt]{article}
\usepackage[papersize={8cm,8cm},margin={.5cm,.5cm}]{geometry}
\usepackage{common}
\usepackage{xeCJKfntef}
\begin{document}
\begin{problem}[widest=10]
  \item[10.] 將甲、乙、丙三個正分數化為最簡分數後,其分子分別為 6、15、10,且其分母的最小公倍數為 360。關於甲、乙、丙三數的大小關係,下列敘述何者正確?
  \begin{choices}
    \item $\textrm{乙} > \textrm{甲} > \textrm{丙}$
    \item $\textrm{乙} > \textrm{丙} > \textrm{甲}$
    \item $\textrm{甲} > \textrm{乙} > \textrm{丙}$
    \item $\textrm{甲} > \textrm{丙} > \textrm{乙}$
  \end{choices}
\end{problem}
\end{document}

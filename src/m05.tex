\documentclass[12pt]{article}
\usepackage[papersize={8cm,14cm},margin={.5cm,.5cm}]{geometry}
\usepackage{common}
\usepackage{xeCJKfntef}
\begin{document}
\begin{problem}
  \small
  \item[5.] \CJKunderline{新世界動物園}準備了刮刮樂,讓開幕當日入園的前 100 位遊客每人皆能索取一張並兌換刮中的獎品。已知獎品種類與數量如表(一)所示,若每張刮刮樂被拿到的機會相同,則開幕當天第一位遊客刮中北極熊玩偶的機率為何?
  \begin{table}[ht]
    \centering\small
    \renewcommand{\arraystretch}{1.2}
    \vspace*{-1ex}
    \caption*{表(一)}
    \vspace*{-2ex}
    \begin{tabular}{|c|c|}
      \hline
      \textbf{獎品} & \textbf{數量} \\ \hline
      北極熊玩偶 & 2 \\ \hline
      造型馬克杯 & 8 \\ \hline
      多功能鑰匙圈 & 20 \\ \hline
      紀念貼紙 & 70 \\ \hline
    \end{tabular}
    \vspace*{-2ex}
  \end{table}
  \begin{choices}
    \item $\fraction{1}{50}$
    \item $\fraction{2}{25}$
    \item $\fraction{1}{5}$
    \item $\fraction{7}{10}$
  \end{choices}
\end{problem}
\end{document}

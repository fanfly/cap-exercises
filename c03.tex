\documentclass[12pt]{article}
\usepackage[papersize={8cm,8cm},margin={.5cm,.5cm}]{geometry}
\usepackage{common}
\begin{document}
\begin{problem}
  \item[3.] 算式 $(\sqrt{18} - \sqrt{8}) \times \sqrt{10}$ 之值為何?
  \begin{choices}
    \item $2\sqrt{5}$
    \item $10$
    \item $4\sqrt{5}$
    \item $20$
  \end{choices}
\end{problem}
\end{document}
